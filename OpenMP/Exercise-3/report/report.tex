\documentclass{article}

% Add 'authorblk' package for affliation support
\usepackage{authblk}

% use package to control paper dimensions
\usepackage{geometry}
 \geometry{
		a4paper,
		% left=22mm,
		% right=22mm,
		top=20mm,
		bottom=20mm
	}
	
% use packages for code listings
\usepackage{listings}
\usepackage{color}

\definecolor{mGreen}{rgb}{0,0.6,0}
\definecolor{mGray}{rgb}{0.5,0.5,0.5}
\definecolor{mPurple}{rgb}{0.58,0,0.82}
\definecolor{backgroundColour}{rgb}{0.95,0.95,0.92}

\lstdefinestyle{CStyle}{
    backgroundcolor=\color{backgroundColour},   
    commentstyle=\color{mGreen},
    keywordstyle=\color{magenta},
    numberstyle=\tiny\color{mGray},
    stringstyle=\color{mPurple},
    basicstyle=\footnotesize,
    breakatwhitespace=false,         
    breaklines=true,                 
    captionpos=b,                    
    keepspaces=true,                 
    showspaces=false,                
    showstringspaces=false,
    showtabs=false,                  
    tabsize=2,
    language=C
}

\title{Analyzing Performance of Spatial-Domain Linear Filtering on Multi-Channel Images using OpenMP}
\author{R Mukesh}
\affil{IIITDM, Kancheepuram}

\date{September 3, 2018}

\begin{document}
	\maketitle
	
	\begin{abstract}
		Typically, Image Processing Algorithms are massively parallelizable and are performed on specialized hardwares such as Graphics Processing units(GPUs) that offer high-throughput. This experiment aims to analyze the performance of parallelized Spatial-Domain linear filtering for multi-channel images using OpenMP.
	\end{abstract}
	
	\section{Theory}
	
		\subsection{Spatial-Domain Linear Filtering}
			\textbf{Linear Filtering} is neighborhood operation, in the which the value of any given pixel in the output image is represented as a linear function of the pixel values in the neighborhood of the corresponding input pixel. Linear filtering in spatial domain is performed using \textbf{convolution}.
			
			\textbf{Mean Filtering} is a linear filtering in which value of any given pixel in the ouput image is the mean of values of pixel in a \(kxk\) window centered at the corresponding input pixel (where k is an odd number).
					
		\subsection{Sections in OpenMP}
			
			The section construct in OpenMP is a way of distributing different independent blocks of codes (each performing a task) to different threads. The sytax of the sections construct is:
			
		\begin{lstlisting}[style=CStyle]
			#pragma omp parallel
			{
				#pragma omp sections
				{
					#pragma omp section
					{
						//structured block 1
					}
					
					#pragma omp section
					{
						//structured block 2
					}
					
					#pragma omp section
					{
						//structured block 3
					}
					
					...
				}
			}
		\end{lstlisting}				
			
	\section{Results \protect\footnote{The experiments were conducted on a computer with 6\textsuperscript{th} Generation Intel(R) Core(TM) i7-6500U Processor (4M Cache, upto 3.10 GHz) and 4GB Single Channel DDR3L 1600M Hz (4GBx1) RAM.}}
	
		The experiments were conducted on a three-channel (red, green and blue) image of size 384x512 for varying size of filter kernels. Each section block performed mean filtering of one channel of the image.
		
	\section{Infererence}
	
	
\end{document}